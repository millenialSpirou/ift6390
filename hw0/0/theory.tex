\section{\enfr{Theoretical Part}{Partie Théorique} \points{5}{5}}
\instruct{%
\enfr{This part of the homework will not be graded for correctness. 
You will still need to provide an answer and properly annotate it on Gradescope. 
You get  1 point for each answer that you properly annotate, even if the answer itself is incorrect.
We will provide you with solutions which you can use to check your answers. 
The goal is for you to
i) practice the process of annotating answers to questions on Gradescope and,
ii) get an idea of what kind of basic math we will need during the class.
If you are having trouble answering these questions you might be missing some important prerequisite knowledge for this class.}%
{ Cette partie du devoir ne sera pas notée.
Vous devez cependant fournir une réponse et l'annotée correctement sur Gradescope.
Vous recevez 1 point pour chaque réponse que vous annotez correctement, même si la réponse n'est pas juste.
Nous vous fournirons des solutions que pourrez utiliser pour vérifier vos réponses.
Le but est que vous puissiez :
i) vous familiariser avec le processus d'annotation de réponses sur Gradescope, et
ii) obtenir une idée du genre de bases mathématiques nécessaires pour ce cours.
Si vous avez des difficulté à répondre à ces questions, il vous manque peut être des pré-requis important pour suivre ce cours.}}
\begin{enumerate}

\item \points{1}{1}
\enfr{Let $X$ be a random variable representing the outcome of a single roll of a 6-sided dice.
Show the steps for the calculation of
i) the expectation of $X$ and
ii) the variance of $X$.}%
{Soit $X$ une variable aléatoire représentant le résultat d'un lancer de dé à 6 faces. Montrez les étapes pour le calcul de 
i) l'espérance de $X$ et
ii) la variance de $X$.}

\item \points{1}{1}
\enfr{Let $u,v\in \mathbb{R}^d$ be two vectors and let $A\in\mathbb{R}^{n\times d}$ be a matrix. Give the formulas for the euclidean norm of $u$, for the euclidean inner product~(aka dot product) between $u$ and $v$, and for the matrix-vector product $Au$.}%
{Soit $u,v\in \mathbb{R}^d$ deux vecteurs et soit $A\in\mathbb{R}^{n\times d}$ une matrice. Donnez la formule pour la norme euclidienne de $u$, pour le produit scalaire euclidien (dot product) de $u$ et $v$, et pour le produit matrice-vecteur $Au$}

\item \points{1}{1}
Consider the two algorithms below. What do they compute and which algorithm is faster? \\
\textit{Observez les deux algorithms ci-dessous. Que calculent-ils et lequel est le plus rapide ?}

\begin{tabular}{ll}
\textbf{ALGO1}(n) \hspace{4cm} & \textbf{ALGO2}(n) \\
\texttt{result = $0$}     &  \texttt{return $(n+1)*n/2$}  \\ 
\texttt{for $i=1\dots n$} & \\
\texttt{\ \ result = result + $i$  } & \\
\texttt{return result} &
\end{tabular}


\item \points{1}{1}
\enfr{Give the step-by-step derivation of the following derivatives:}%
{Donnez les étapes de calculs des dérivées suivantes:}
$$
  \textrm{i)} \quad \frac{df}{dx}=?, \quad \textrm{where}\  f(x,\beta) = x^2 \exp{(-\beta x)} 
$$
$$
  \textrm{ii)} \quad \frac{df}{d\beta}=?, \quad \textrm{where}\  f(x,\beta) = x \exp{(-\beta x)} 
$$
$$
  \textrm{iii)} \quad \frac{df}{dx}=?, \quad \textrm{where}\  f(x) = \sin{(\exp{(x^2)})} 
$$


\item \points{1}{1}
\enfr{Let $X\sim N(\mu,1)$, that is the random variable $X$ is distributed according to a Gaussian with mean $\mu$ and standard deviation $1$.
Show how you can calculate the second moment of $X$, given by $\mathbb{E}[X^2]$.}%
{Soit $X\sim N(\mu,1)$, c'est à dire que $X$ est une variable aléatoire suivant une distribution gaussienne de moyenne $\mu$ et d'écart-type $1$. Montrez comment calculer le second moment de $X$, donné par $\mathbb{E}[X^2]$.}

\end{enumerate}


\clearpage

1. $X$ is a multinoulli random variable. Hence it is a vector of 
length 6 where by convention we say the ith entry is 1 if the dice
roles i and 0 elsewhere. We assume the dice is fair each entry has
probability 1/6. The expected value is thus a vector with 1/6 in
each entry. 
\begin{align}
	P (X_i = 1) &= 1/6\\
	E X_i = 1*1/6 + 0*5/6 &= 1/6
\end{align}

Since X is a vector its variance is actually a 6 by 6 covariance matrix:

\begin{equation}
  Cov[X_i,X_j] = E[X_i X_j] - E[X_i]E[X_j]
\end{equation}

If $i=j$ we have

\begin{align}
  Cov [X_i ,X_j] = E[X_{i}^{2}] - {E[X_{i}]}^{2} &= p_{i}-p_{i}^{2}\\
                                               &= 1/16 - 1/16^2\\
                                               &= 15/256
\end{align}
Where we use $X_{i}^{2} = X_{i}$ since $X_{i}$ is this random variable can only take
values 1 and 0.

Now if $i \neq j$ we have that the product $X_{i}$ and $X_{j}$ is zero since
only one of the entries of $X$ can be non zero at a time, hence:

\begin{align}
  Cov [X_i ,X_j] = E[X_{i} X_{j}] - E[X_i] E[X_j] &= -E[X_i]E[X_j]\\
                                                  &= -p_i ^ 2 \\
                                                  &= -1/256
\end{align}


\clearpage

2.

\begin{enumerate}
  \item euclidean norm:
    $\| u \| =\sqrt{\sum_{i=1}^{d} u_{i}^{2}} $
  \item inner product:
    $\langle u, v \rangle = \sum_{i=1}^{d} u_{i}v_{i}$
  \item matrix product:
    $(Au)_k = \langle A_{k,:} , u \rangle $
    where $(Au)_k $ denotes the kth entry in the resulting vector.

\end{enumerate}

3.

Both algorithms compute the sum of the first n natural numbers. The second
algorithm is faster because it computes this sum with an expression that
contains a number of arithmetic operations constant with respect to the size of
the n, i.e. 3 arithmetic operation. The first algorithm on the other hand
computes the result in a number of arithmetic operations linear in the size of
n.                                           .

4.

\begin{enumerate}
  \item $\frac{df}{dx} = -\beta x^{2} \exp(-\beta x)$
  \item $\frac{df}{d\beta} = -x^{2} \exp (-\beta x)$
  \item $\frac{df}{x} = \cos (\exp(x^{2})) \sin(2x \exp(x^{2}))$
\end{enumerate}


5.

$E[X^2]$ can be computed with the following formula:
\begin{align}
  Var(X) &= E[(X-E[X])^2] \\
         &= E[X^2 - 2XE[X] + E[X^2]]\\
         &= E[X^2] - 2E[X]E[X] + E[X^2]\\
         &= E[X^{2}] - E[X]^{2}
\end{align}

Hence in our case $E[X^{2}] = Var(X) + E[X]^{2}= 1 + \mu^{2}$


\clearpage
